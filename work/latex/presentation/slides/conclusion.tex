
\section{Conclusion \& Perspectives}

\begin{frame}{\underline{\secname}}

\begin{itemize}\setlength\itemsep{0.5em}
	\item  Les Principes de la Thermodynamique gouvernent le
tout, même les trous noirs !

 \item L’idée de l’identification de la constante cosmologique avec la pression donne une façon de définir le volume des trous noirs

\item Un trou noir de Schwarzschild AdS peut donner une transition de phase de type HP.

\item Les trous noirs RN-AdS et Kerr-AdS présentent un comportement thermodynamique similaire à celui d’un fluide de Van-der-Wals


\end{itemize}.

\textbf{L'étape suivante:}

\begin{itemize}\setlength\itemsep{0.5em}
	\item Utilier la géométrie thermodynamique pour étudier le comportement thermodynamique des trous noirs.
	\item La thermodynamique du trou noir à D-dimention (avec $D > 4$)
\end{itemize}
\end{frame}