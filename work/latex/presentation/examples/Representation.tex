%\section{Definition}
% --------------------------------------------------- Slide --
%\subsection{Definition}
\label{definition}
\begin{frame}[t]\frametitle{Dirac basis}

in the Dirac basis:\newline


\begin{center}
	$\gamma ^{0}={\begin{pmatrix}I_{2}&0\\0&-I_{2}\end{pmatrix}},\quad \gamma ^{k}={\begin{pmatrix}0&\sigma ^{k}\\-\sigma ^{k}&0\end{pmatrix}},\quad \gamma ^{5}={\begin{pmatrix}0&I_{2}\\I_{2}&0\end{pmatrix}}$
 
\end{center}
\newline
 \newline
\begin{center}
	 
$( P^\mu P_\mu - m^2 )=( \gamma^k P_k + m)(\gamma_k P^k  -m) = 0$ 
 
\end{center}
 
\begin{itemize}
	 \item Dirac equation $\equiv$ K-G equation
 \item In bass energy
  \item  Appropriate for acting on Dirac spinors
\end{itemize}

\end{frame}
\begin{frame}[t]\frametitle{Majorana basis}
in the Majorana basis:\\

\newline
\begin{center}
	${\displaystyle {\begin{aligned}\gamma ^{0}&={\begin{pmatrix}0&\sigma ^{2}\\\sigma ^{2}&0\end{pmatrix}},&\gamma ^{1}&={\begin{pmatrix}i\sigma ^{3}&0\\0&i\sigma ^{3}\end{pmatrix}},&\gamma ^{2}&={\begin{pmatrix}0&-\sigma ^{2}\\\sigma ^{2}&0\end{pmatrix}},\\\gamma ^{3}&={\begin{pmatrix}-i\sigma ^{1}&0\\0&-i\sigma ^{1}\end{pmatrix}},&\gamma ^{5}&={\begin{pmatrix}\sigma ^{2}&0\\0&-\sigma ^{2}\end{pmatrix}}\end{aligned}}}$
\end{center}
\newline

\begin{itemize}
 \item  Positive energie for particule and anti-particule
 \item  Appropriate for Majorana fermion  
\item  A Majorana fermion is defined as a fermion that equals to its charge conjugation
\end{itemize}

\end{frame}
\begin{frame}[t]\frametitle{ Weyl (chiral) basis }
% 
% Another common choice is the Weyl or chiral basis,[5] in which ${\displaystyle \gamma ^{k}}\gamma ^{k}$ remains the same but ${\displaystyle \gamma ^{0}}\gamma ^{0}$ is different, and so ${\displaystyle \gamma ^{5}}\gamma ^{5}$ is also different, and diagonal,

in the Weyl basis:\newline

\begin{center}
	
$\gamma ^{0}={\begin{pmatrix}0&I_{2}\\I_{2}&0\end{pmatrix}},\quad \gamma ^{k}={\begin{pmatrix}0&\sigma ^{k}\\-\sigma ^{k}&0\end{pmatrix}},\quad \gamma ^{5}={\begin{pmatrix}-I_{2}&0\\0&I_{2}\end{pmatrix}}$

\end{center}
\newline

\begin{center}
	${\displaystyle \psi _{L}={\frac {1}{2}}\left(1-\gamma ^{5}\right)\psi ={\begin{pmatrix}I_{2}&0\\0&0\end{pmatrix}}\psi ,\quad \psi _{R}={\frac {1}{2}}\left(1+\gamma ^{5}\right)\psi ={\begin{pmatrix}0&0\\0&I_{2}\end{pmatrix}}\psi }$
\end{center}


\begin{itemize}
 \item  The Weyl basis has the advantage that its chiral projections take a simple form,
 \item In high energy
\end{itemize}



\end{frame}
