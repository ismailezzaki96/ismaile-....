%\section{Columns}
% --------------------------------------------------- Slide --
%\subsection{Columns}
\label{columns}
\begin{frame}\frametitle{Demonstration of Clifford’s rule }

\begin{itemize}
	\item 
Clifford apply a general definition of product given by Gressman : “\textbf{a product of two extensives quantities u and v is defined as extensive quantity or a scalar}.”


	\item 

Clifford consider a n-fundamental units $(e_1,e_2,...,e_n)$ \\

\begin{center}
	$(e_j)^2=+1$or $-1$  and $e_j.e_k=-e_k.e_j (j \neq k)$ 
\end{center}
	\item 
%where e denotes the identity. \\
All product of linear factors : \\odd order $(ej, e_j e_k e_l, ... $such that $j \neq k \neq l)$,
\\or  even order $( 1, e_j e_k, ...$ such that $j \neq k)$. \\

%There can be no such term of higher order than the n. 

	\item 
The total number of basic terms is $2^n$ units.


\end{itemize}

\end{frame}


\begin{frame}\frametitle{Demonstration of Clifford’s rule }

The product of all $n-$units:\\
\begin{center}
	$w \equiv e_1 e_2...e_n$ \\
$w^2 \equiv e_1 e_2....e_n e_1 e_2...e_n$,\\

\end{center}
After $r-$interchanges becomes:
\[
w^2 =  (-1)^r (e_{1})^2 (e_{2})^2 ...(e_{n})^2 
\begin{cases}
(-1)^r,& \text{if } (e_j)^2 = + 1\\
(-1)^{r+n} &  \text{if } (e_j)^2 = - 1
\end{cases}


\]

\end{frame}



\begin{frame}\frametitle{Demonstration of Clifford’s rule }

 Next, he considers the nature of the multiplication w ej' He observes that:\\
\[
   e_j w = 
\begin{cases}
+ e_1 e_2 e_3 ...e_n,& \text{if } (e_j)^2 = + 1\\
- e_1 e_2 e_3 ...e_n,              &  \text{if } (e_j)^2 = - 1
\end{cases}
\\ 
\text{and }\\
    w e_j= 
\begin{cases}
(1)^n e_1 e_2 e_3 ...e_n,& \text{if } (e_j)^2 = + 1\\
(1)^{n-1} e_1 e_2 e_3 ...e_n,              &  \text{if } (e_j)^2 = - 1
\end{cases}

\]

 
\textbf{ $w e_j $ is commutative or anti-commutative }
 

Therefore, we can put :\\
\begin{center}
	$w \equiv \sqrt{-1}$ 
\end{center}
\end{frame}


\begin{frame}\frametitle{Demonstration of Clifford’s rule }


he considers the factors of units of even order

\begin{center}
	$1, e_j e_k, e_j e_k e_i e_m...$ \text{such that } $j \neq k \neq I$;
\end{center}
they form an algebra by themselves which he calls the
\begin{center}
	 {\color{red} Even subalgebra}
\end{center} 

\textbf{Example}: the even subalgebra with basic elements 
\begin{center}
	$\{I; e_2 e_3; e_l e_3; e_l e_2\}$
\end{center}
 gives the quaternions with $i = e_2 e_3\;\;\;,\;\;\; j = e_l e_3\;\;\;, \;\;\;k = e_1 e_2$





\end{frame}
\begin{frame}\frametitle{Demonstration of Clifford’s rule }

The general element of the even subalgebra :\\

\begin{center}
	$E = q_0 1 + \sum_{j\neq K=1}^{3} q_{jk} e_j e_k +  q_0' w$
\end{center}


%where $q$ and $q'$ are quaternions. \\

And we add the quadratic form :\\

\begin{center}
	$(\sum_{j=1}^{n} q_j e_j )^2 = \sum_{j=1}^{n} q_{j}^2$
\end{center}



we get Clifford's rule \\
{\color{red}
\textbf{\begin{center}
	$e_j e_k + e_k e_j = 2 \delta_{jk} 1 \;\;,\;\; (j  ,k=I, ... ,n).$
\end{center}}}


\end{frame}