\chaptertoc{Introduction Générale}

Contrairement à ce que l'on pense généralement de la physique nucléaire, les propriétés des noyaux ne sont pas encore totalement comprises et le domaine a récemment accompli des progrès majeurs tant d'un point de vue théorique qu'expérimental.

Auparavant, l'étude des noyaux atomiques se limitait aux noyaux stables ou à ceux situés près de la vallée de stabilité. Cependant, au fil des années, Les recherches en physique nucléaire ont été étendues aux noyaux exotiques. Les modèles nucléaires, qui sont essentiellement basés sur des noyx proches de stabilité, divergent à mesure que l'on s'approche des limites de Stabilité. Pour remédier à ce problème plusieurs approches ont été élaborées  pour décrire les noyaux stables et appliquées à certains noyaux exotiques.

Pour les noyaux légers, le nombre relativement petit de nucléons donne la possibilité d’utiliser l’interaction à deux corps pour reproduire les comportements nucléaires, par exemple : calculs ab-initio (fonction de Green - modèle en couches Monte-Carlo). Les noyaux de masse moyenne jusqu’à A ${\sim}$ 60 peuvent être traités par le modèle en couche à grande échelle. Pour les noyaux plus lourds, on utilise des théories de champ moyen soient non relativistes  ou relativistes. 

Les méthodes microscopiques utilisent les théories du champ moyen ont acquis un degré de fiabilité remarquable car ils sont moins compliqués et ils décrivent avec succès les propriétés statiques et dynamiques des noyaux. L'une des approches phénoménologiques les plus importantes, largement utilisée dans les calculs de structure nucléaire, est la méthode Hartree-Fock-Bogoliubov, qui permet de traiter les particules et les trous sur un pied d'égalité en unifiant la description auto-cohérente des orbitales nucléaires, telle qu'elle est donnée par l'approche Hartree-Fock (HF), et la théorie de l'appariement Bardeen-Cooper-Schrieffer (BCS) en une seule méthode variationnelle.

L’objectif de ce travail est d’étudier les propriétés de l’état fondamental de plusieurs chaines isotopiques, allant du côté riche en protons jusqu’au côté riche en neutrons, en utilisant la méthode de Hartree Fock Bogoliubov non relativiste.


\thispagestyle{empty}

